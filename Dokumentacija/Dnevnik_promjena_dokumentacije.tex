\chapter{Dnevnik promjena dokumentacije}
		
		\textbf{\textit{Kontinuirano osvježavanje}}\\
				
		
		\begin{longtblr}[
				label=none
			]{
				width = \textwidth, 
				colspec={|X[2]|X[13]|X[3]|X[3]|}, 
				rowhead = 1
			}
			\hline
			\textbf{Rev.}	& \textbf{Opis promjene/dodatka} & \textbf{Autori} & \textbf{Datum}\\[3pt] \hline
			0.1 & Napravljen predložak.	& Mihael Miličević & 19.10.2021.		\\[3pt] \hline
			0.2 & Dodani dionici u projektu, te aktori i njihovi funkcionalni zahtjevi.	& Mihael Miličević & 20.10.2021.		\\[3pt] \hline 
			0.3 & Dodani obrasci uporabe.	& Mihael Miličević, Tomislav Žiger & 24.10.2021.		\\[3pt] \hline 
			0.4 & Dodan opis projektnog zadatka.	& Mihael Miličević & 24.10.2021.		\\[3pt] \hline 
			0.4.1 & Dodani ostali zahtjevi.	& Tomislav Žiger & 27.10.2021.		\\[3pt] \hline 
			0.4.2 & Ispravak obrazaca uporabe.	& Mihael Miličević & 28.10.2021.		\\[3pt] \hline 
			0.4.3 & Dodani dijagrami obrazaca uporabe.	& Mihael Miličević, Tomislav Žiger & 3.11.2021.		\\[3pt] \hline 
			0.4.4 & Dodani sekvencijski dijagrami.	& Mihael Miličević, Tomislav Žiger & 3.11.2021.		\\[3pt] \hline 
			0.5 & Dodana specifikacija baze podataka.	& Andrija Banić & 3.11.2021.	
			% \textbf{1.0} & Verzija samo s bitnim dijelovima za 1. ciklus & * & 11.09.2013. \\[3pt] \hline 
			% \textbf{2.0} & Konačni tekst predloška dokumentacije  & * & 28.09.2013. \\[3pt] \hline 
			% &  &  & \\[3pt] \hline	
		\end{longtblr}
	
	
		\textit{Moraju postojati glavne revizije dokumenata 1.0 i 2.0 na kraju prvog i drugog ciklusa. Između tih revizija mogu postojati manje revizije već prema tome kako se dokument bude nadopunjavao. Očekuje se da nakon svake značajnije promjene (dodatka, izmjene, uklanjanja dijelova teksta i popratnih grafičkih sadržaja) dokumenta se to zabilježi kao revizija. Npr., revizije unutar prvog ciklusa će imati oznake 0.1, 0.2, …, 0.9, 0.10, 0.11.. sve do konačne revizije prvog ciklusa 1.0. U drugom ciklusu se nastavlja s revizijama 1.1, 1.2, itd.}