\chapter{Specifikacija programske potpore}
		
	\section{Funkcionalni zahtjevi}
			
			\textbf{\textit{dio 1. revizije}}\\
			
			\textit{Navesti \textbf{dionike} koji imaju \textbf{interes u ovom sustavu} ili  \textbf{su nositelji odgovornosti}. To su prije svega korisnici, ali i administratori sustava, naručitelji, razvojni tim.}\\
				
			\textit{Navesti \textbf{aktore} koji izravno \textbf{koriste} ili \textbf{komuniciraju sa sustavom}. Oni mogu imati inicijatorsku ulogu, tj. započinju određene procese u sustavu ili samo sudioničku ulogu, tj. obavljaju određeni posao. Za svakog aktora navesti funkcionalne zahtjeve koji se na njega odnose.}\\
			
			
			\noindent \textbf{Dionici:}
			
			\begin{packed_enum}
				
				\item Vlasnik (naručitelj)
				\item Stanovnik četvrti			
				\begin{packed_enum}
					\item  Obični stanovnik
					\item  Vijećnik
					\item  Moderator
				\end{packed_enum}
				\item Administrator	
				\item Razvojni tim
				
			\end{packed_enum}
			
			\noindent \textbf{Aktori i njihovi funkcionalni zahtjevi:}
			
			
			\begin{packed_enum}
			
				\item  \underbar{Neregistrirani korisnik (inicijator) može:}
				\begin{packed_enum}
					\item se registrirati u sustav stvaranjem novog korisničkog računa, za što su mu potrebni ime, prezime, adresa stanovanja, adresa e-pošte i lozinka
				\end{packed_enum}
				
				\item  \underbar{Neprijavljeni korisnik (inicijator) može:}
				\begin{packed_enum}
					\item se prijaviti u sustav,	za što su mu potrebni adresa e-pošte i lozinka
				\end{packed_enum}
				
				\item  \underbar{Stanovnik četvrti (inicijator) može:}
				\begin{packed_enum}
					\item čitati teme na "Forumu"
					\item otvarati teme na "Forumu"
					\item odgovarati u temama na "Forumu"
					\item uređivati svoje odgovore na "Forumu"
					\item ukloniti svoje odgovore na "Forumu"
					\item otvoriti temu na "Forumu" vezanu za objavu na "Vijeću četvrti", ako za tu objavu već nije otvorena tema
					\item čitati objave na "Vijeću četvrti"
					\item vidjeti najave događanja u cjelini "Događanja"
					\item predlagati najave budućih događanja u cjelini "Događanja", za što je potrebno navesti naziv, mjesto, vrijeme, trajanje i kratak opis
					\item poslati zahtjev administratorima za promjenu uloge (u vijećnika ili moderatora)
					\item promijeniti adresu	
					\item odjaviti se iz sustava
				\end{packed_enum}
				
				\item  \underbar{Vijećnik (inicijator) može:}
				\begin{packed_enum}
					\item stvoriti objavu u cjelini "Vijeće četvrti"	
				\end{packed_enum}
				
				\item  \underbar{Moderator (inicijator) može:}
				\begin{packed_enum}
					\item pregledati najave događanja koje predlažu stanovnici
					\item objaviti najave događanja u cjelini "Događanja"
					\item odbaciti prijedloge događanja
					\item urediti prijedloge događanja
					\item ukloniti odgovore u temama na "Forumu"
					\item ukloniti teme na Forumu
				\end{packed_enum}
				
				\item  \underbar{Administrator (inicijator) može:}
				\begin{packed_enum}
					\item vidjeti popis svih registriranih korisnika i njihove osobne podatke
					\item definirati četvrt i područje koje ta četvrt obuhvaća
					\item obrisati četvrti
					\item dodijeliti ulogu moderatora ili vijećnika registriranom korisniku
					\item privremeno blokirati korisnika
					\item trajno obrisati profil korisnika
					\item pristupiti svim podacima u bazi podataka (teme na forumima, događanja u četvrtima i slično)	
				\end{packed_enum}
				
				\item  \underbar{Baza podataka (sudionik):}
				\begin{packed_enum}
					\item pohranjuje podatke o korisnicima i njihovim ovlastima
					\item pohranjuje podatke o četvrtima
					\begin{packed_enum}
						\item ulice koje im pripadaju
						\item teme na "Forumu"
						\item izvješća s "Vijeća četvrti"
					\end{packed_enum}
				\end{packed_enum}
				
			\end{packed_enum}
			
			\eject 
			
			
				
			\subsection{Obrasci uporabe}
				
				\textbf{\textit{dio 1. revizije}}
				
				\subsubsection{Opis obrazaca uporabe}
					\textit{Funkcionalne zahtjeve razraditi u obliku obrazaca uporabe. Svaki obrazac je potrebno razraditi prema donjem predlošku. Ukoliko u nekom koraku može doći do odstupanja, potrebno je to odstupanje opisati i po mogućnosti ponuditi rješenje kojim bi se tijek obrasca vratio na osnovni tijek.}\\
					

					\noindent \underbar{\textbf{UC$<$broj obrasca$>$ -$<$ime obrasca$>$}}
					\begin{packed_item}
	
						\item \textbf{Glavni sudionik: }$<$sudionik$>$
						\item  \textbf{Cilj:} $<$cilj$>$
						\item  \textbf{Sudionici:} $<$sudionici$>$
						\item  \textbf{Preduvjet:} $<$preduvjet$>$
						\item  \textbf{Opis osnovnog tijeka:}
						
						\item[] \begin{packed_enum}
	
							\item $<$opis korak jedan$>$
							\item $<$opis korak dva$>$
							\item $<$opis korak tri$>$
							\item $<$opis korak četiri$>$
							\item $<$opis korak pet$>$
						\end{packed_enum}
						
						\item  \textbf{Opis mogućih odstupanja:}
						
						\item[] \begin{packed_item}
	
							\item[2.a] $<$opis mogućeg scenarija odstupanja u koraku 2$>$
							\item[] \begin{packed_enum}
								
								\item $<$opis rješenja mogućeg scenarija korak 1$>$
								\item $<$opis rješenja mogućeg scenarija korak 2$>$
								
							\end{packed_enum}
							\item[2.b] $<$opis mogućeg scenarija odstupanja u koraku 2$>$
							\item[3.a] $<$opis mogućeg scenarija odstupanja  u koraku 3$>$
							
						\end{packed_item}
					\end{packed_item}
					
					\noindent \underbar{\textbf{UC1 - Registracija}}
					\begin{packed_item}
	
						\item \textbf{Glavni sudionik: }Korisnik
						\item  \textbf{Cilj:} Stvoriti korisnički račun za pristup sustavu
						\item  \textbf{Sudionici:} Baza podataka
						\item  \textbf{Preduvjet:} -
						\item  \textbf{Opis osnovnog tijeka:}
						
						\item[] \begin{packed_enum}
	
							\item Korisnik odabere opciju za registraciju
							\item Korisnik uspješno unese tražene podatke
							\item Korisnik dobiva obavijest o uspješnoj registraciji
							\item Korisnik biva preusmjeren na početnu stranicu
						\end{packed_enum}
						
						\item  \textbf{Opis mogućih odstupanja:}
						
						\item[] \begin{packed_item}
	
							\item[2.a] Korisnik odabere e-mail s kojim je već povezan korisnički račun u sustavu, ili unese tražene podatke u krivom formatu
							\item[] \begin{packed_enum}
								
								\item Korisnik dobiva obavijest o pogrešci
								\item Korisnik mijenja podatke i ponovno odabire opciju registracije, ili odustaje od registracije
								
							\end{packed_enum}
							
						\end{packed_item}
					\end{packed_item}
					
					\noindent \underbar{\textbf{UC2 - Prijava u sustav}}
					\begin{packed_item}
	
						\item \textbf{Glavni sudionik: }Korisnik
						\item  \textbf{Cilj:} Dobiti pristup sustavu
						\item  \textbf{Sudionici:} Baza podataka
						\item  \textbf{Preduvjet:} -
						\item  \textbf{Opis osnovnog tijeka:}
						
						\item[] \begin{packed_enum}
	
							\item Korisnik odabire opciju prijave u sustav
							\item Korisnik unosi e-mail adresu i lozinku
							\item Korisnik dobiva pristup sustavu
							\item Korisnik biva preusmjeren na početnu stranicu
						\end{packed_enum}
						
						\item  \textbf{Opis mogućih odstupanja:}
						
						\item[] \begin{packed_item}
	
							\item[3.a] Korisnik unosi e-mail s kojim nije povezan niti jedan korisnički račun, ili unosi krivu lozinku, ili unosi tražene podatke u krivom formatu
							\item[] \begin{packed_enum}
								
								\item Korisnik dobiva obavijest o pogrešci
								\item Korisnik mijenja podatke i ponovno odabire opciju prijave, ili odustaje od prijave
								
							\end{packed_enum}
							
						\end{packed_item}
					\end{packed_item}
					
					\noindent \underbar{\textbf{UC3 - Pregled osobnih podataka}}
					\begin{packed_item}
	
						\item \textbf{Glavni sudionik: }Korisnik
						\item  \textbf{Cilj:} Vidjeti osobne podatke
						\item  \textbf{Sudionici:} Baza podataka
						\item  \textbf{Preduvjet:} Korisnik je prijavljen u sustav
						\item  \textbf{Opis osnovnog tijeka:}
						
						\item[] \begin{packed_enum}
	
							\item Korisnik odabire opciju "Osobni podaci"
							\item Korisnik dobije prikaz osobnih podataka
						\end{packed_enum}
					\end{packed_item}
					
					\noindent \underbar{\textbf{UC4 - Promjena osobnih podataka}}
					\begin{packed_item}
	
						\item \textbf{Glavni sudionik: }Korisnik
						\item  \textbf{Cilj:} Promijeniti osobne podatke
						\item  \textbf{Sudionici:} Baza podataka
						\item  \textbf{Preduvjet:} Korisnik je prijavljen u sustav
						\item  \textbf{Opis osnovnog tijeka:}
						
						\item[] \begin{packed_enum}
	
							\item Korisnik odabire opciju promjene osobnih podataka
							\item Korisnik mijenja svoje osobne podatke
							\item Korisnik sprema promjene
							\item Baza podataka se ažurira
						\end{packed_enum}
						
						\item  \textbf{Opis mogućih odstupanja:}
						
						\item[] \begin{packed_item}
	
							\item[2.a] Korisnik mijenja svoje osobne podatke, ali ne spremi promjene
							\item[] \begin{packed_enum}
								
								\item Sustav upozori korisnika da nije spremio promjene
								
							\end{packed_enum}
							
							\item[2.b] Korisnik pokuša promijeniti e-mail adresu
							\item[] \begin{packed_enum}
								
								\item Sustav upozori korisnika da ne može promijeniti e-mail adresu
								
							\end{packed_enum}
							
						\end{packed_item}
					\end{packed_item}
					
					\noindent \underbar{\textbf{UC5 - Brisanje korisničkog računa}}
					\begin{packed_item}
	
						\item \textbf{Glavni sudionik: }Korisnik
						\item  \textbf{Cilj:} Obrisati korisnički račun
						\item  \textbf{Sudionici:} Baza podataka
						\item  \textbf{Preduvjet:} Korisnik je prijavljen u sustav
						\item  \textbf{Opis osnovnog tijeka:}
						
						\item[] \begin{packed_enum}
	
							\item Korisnik pregledava korisničke podatke
							\item Korisnik bira opciju brisanja korisničkog računa
							\item Korisnički račun se izbriše iz baze podataka
							\item Korisnik biva preusmjeren na stranicu za prijavu
							
						\end{packed_enum}
							
					\end{packed_item}

					\noindent \underbar{\textbf{UC6 - Pregled događaja u cjelini "Događanja" }}
					\begin{packed_item}
	
						\item \textbf{Glavni sudionik: }Korisnik
						\item  \textbf{Cilj:} Prikaz dogovorenih događaja u korisnikovoj četvrti
						\item  \textbf{Sudionici:} Baza podataka
						\item  \textbf{Preduvjet:} Korisnik je prijavljen u sustav
						\item  \textbf{Opis osnovnog tijeka:}
						
						\item[] \begin{packed_enum}
	
							\item Korisnik odabere opciju događaji
							\item Aplikacija prikazuje događaje za njegovu četvrt
						\end{packed_enum}
						
							
						\end{packed_item}
					\noindent \underbar{\textbf{UC7 - Prijedlog događaja}}
					\begin{packed_item}
	
						\item \textbf{Glavni sudionik: }Korisnik
						\item  \textbf{Cilj:} Stvaranje prijedloga korisnikovog događaja
						\item  \textbf{Sudionici:} Baza podataka
						\item  \textbf{Preduvjet:} Korisnik je prijavljen u sustav
						\item  \textbf{Opis osnovnog tijeka:}
						
						\item[] \begin{packed_enum}
	
							\item Korisnik odabire opciju "Novi prijedlog događaja" 
							\item Korisnik ispunjava polja "naziv", "mjesto", "vrijeme", "trajanje" i "kratki opis"
							\item Korisnik stvara novi prijedlog događaja
							\item Baza podataka se ažurira
						\end{packed_enum}
						
						\item  \textbf{Opis mogućih odstupanja:}
						
						\item[] \begin{packed_item}
	
							\item[2.a] Korisnik nije ispunio sva polja prilikom stvaranja prijedloga događaja
							\item[] \begin{packed_enum}
								
								\item Sustav upozori korisnika da sva polja nisu ispunjena
								
							\end{packed_enum}
							
							
						\end{packed_item}
					\end{packed_item}						
					\noindent \underbar{\textbf{UC8 - Pregled prijedloga događaja}}
					\begin{packed_item}
	
						\item \textbf{Glavni sudionik: }Moderator
						\item  \textbf{Cilj:} Pregled prijedloga događaja u cjelini "Događaji"
						\item  \textbf{Sudionici:} Baza podataka
						\item  \textbf{Preduvjet:} Moderatorske ovlasti
						\item  \textbf{Opis osnovnog tijeka:}
						
						\item[] \begin{packed_enum}
	
							\item Moderator odabire u cjelini "Događaji" opciju "Prijedlozi događaja" 
							\item Prikazuju se svi prijedlozi događaja za moderatorovu četvrt
							
						\end{packed_enum}
						\end{packed_item}
						\noindent \underbar{\textbf{UC9 - Uređivanje prijedloga događaja}}
					\begin{packed_item}
	
						\item \textbf{Glavni sudionik: }Moderator
						\item  \textbf{Cilj:} Uređivanje prijedloga događaja u cjelini "Događaji"
						\item  \textbf{Sudionici:} Baza podataka
						\item  \textbf{Preduvjet:} Moderatorske ovlasti
						\item  \textbf{Opis osnovnog tijeka:}
						
						\item[] \begin{packed_enum}
	
							\item Moderator odabire prijedlog događaja 
							\item Moderator uređuje polja za odabrani prijedlog događaja
							\item Moderator sprema promjene
							\item Baza podataka se ažurira
							
						\end{packed_enum}
						\end{packed_item}
					\noindent \underbar{\textbf{UC10 - Objava prijedloga događaja}}
					\begin{packed_item}
	
						\item \textbf{Glavni sudionik: }Moderator
						\item  \textbf{Cilj:} Objava preloženog događaja u cjelini "Događaji"
						\item  \textbf{Sudionici:} Baza podataka
						\item  \textbf{Preduvjet:} Moderatorske ovlasti
						\item  \textbf{Opis osnovnog tijeka:}
						
						\item[] \begin{packed_enum}
	
							\item Moderator odabire prijedlog događaja 
							\item Moderator objavljuje događaj u cjelini "Događanja"
							\item Baza podataka se ažurira
							
						\end{packed_enum}
						\end{packed_item}
					\noindent \underbar{\textbf{UC11 - Brisanje prijedloga događaja}}
					\begin{packed_item}
	
						\item \textbf{Glavni sudionik: }Moderator
						\item  \textbf{Cilj:} Brisanje prijedloga događaja u cjelini "Događaji"
						\item  \textbf{Sudionici:} Baza podataka
						\item  \textbf{Preduvjet:} Moderatorske ovlasti
						\item  \textbf{Opis osnovnog tijeka:}
						
						\item[] \begin{packed_enum}
	
							\item Moderator odabire prijedlog događaja 
							\item Moderator briše prijedlog događaja 
							\item Baza podataka se ažurira
							
						\end{packed_enum}
						
					\end{packed_item}
					\noindent \underbar{\textbf{UC12 - Pregled foruma}}
					\begin{packed_item}
	
						\item \textbf{Glavni sudionik: }Korisnik
						\item  \textbf{Cilj:} Prikaz tema na forumu
						\item  \textbf{Sudionici:} Baza podataka
						\item  \textbf{Preduvjet:} Korisnik je prijavljen u sustav
						\item  \textbf{Opis osnovnog tijeka:}
						
						\item[] \begin{packed_enum}
	
							\item Korisnik odabire cjelinu „Forum“
							\item Prikazuju se teme na forumu poredane po vremenu zadnjeg odgovora
							
						\end{packed_enum}
						
						
					\end{packed_item}						
					\noindent \underbar{\textbf{UC13 - Prikaz teme na forumu}}
					\begin{packed_item}
	
						\item \textbf{Glavni sudionik: }Korisnik
						\item  \textbf{Cilj:} Prikaz željene teme na forumu
						\item  \textbf{Sudionici:} Baza podataka
						\item  \textbf{Preduvjet:} Korisnik je prijavljen u sustav
						\item  \textbf{Opis osnovnog tijeka:}
						
						\item[] \begin{packed_enum}
	
							\item Korisnik u cjelini "Forum" odabire željenu temu
							\item Prikazuje se odabrana tema
							
						\end{packed_enum}
						
						
					\end{packed_item}
					\noindent \underbar{\textbf{UC14 - Odabir objave za komentar }}
					\begin{packed_item}
	
						\item \textbf{Glavni sudionik: }Korisnik
						\item  \textbf{Cilj:} Odabir objave na koji korisnik želi odgovoriti
						\item  \textbf{Sudionici:} Baza podataka
						\item  \textbf{Preduvjet:} Korisnik je prijavljen u sustav
						\item  \textbf{Opis osnovnog tijeka:}
						
						\item[] \begin{packed_enum}
	
							\item Korisnik unutar teme traži objavu na koju želi odgovoriti
							\item Odabire za tu objavu opciju "Odgovori"
							\item Otvara se polje za odgovor koje odgovara na željenu objavu
							
							
						\end{packed_enum}
						
						
					\end{packed_item}
					\noindent \underbar{\textbf{UC15 - Uređivanje odgovora }}
					\begin{packed_item}
	
						\item \textbf{Glavni sudionik: }Korisnik
						\item  \textbf{Cilj:} Uređivanje odgovora u polju za odgovor 
						\item  \textbf{Sudionici:} Baza podataka
						\item  \textbf{Preduvjet:} Korisnik je prijavljen u sustav
						\item  \textbf{Opis osnovnog tijeka:}
						
						\item[] \begin{packed_enum}
	
							\item Korisnik u polju za odgovore uređuje tekst
						\end{packed_enum}
							
						\item  \textbf{Opis mogućih odstupanja:}
						
						\item[] \begin{packed_item}
						\item[2.a] Korisnik napušta stranicu iako napisani odgovor nije objavio
							\item[] \begin{packed_enum}
								
								\item Sustav upozorava korisnika da uređeni tekst neće biti spremljen ako napusti stranicu
								
							\end{packed_enum}
						\end{packed_item}
						
					\end{packed_item}						
										
					
				\subsubsection{Dijagrami obrazaca uporabe}
					
					\textit{Prikazati odnos aktora i obrazaca uporabe odgovarajućim UML dijagramom. Nije nužno nacrtati sve na jednom dijagramu. Modelirati po razinama apstrakcije i skupovima srodnih funkcionalnosti.}
				\eject		
				
			\subsection{Sekvencijski dijagrami}
				
				\textbf{\textit{dio 1. revizije}}\\
				
				\textit{Nacrtati sekvencijske dijagrame koji modeliraju najvažnije dijelove sustava (max. 4 dijagrama). Ukoliko postoji nedoumica oko odabira, razjasniti s asistentom. Uz svaki dijagram napisati detaljni opis dijagrama.}
				\eject
	
		\section{Ostali zahtjevi}
		
			\textbf{\textit{dio 1. revizije}}\\
		 
			 \textit{Nefunkcionalni zahtjevi i zahtjevi domene primjene dopunjuju funkcionalne zahtjeve. Oni opisuju \textbf{kako se sustav treba ponašati} i koja \textbf{ograničenja} treba poštivati (performanse, korisničko iskustvo, pouzdanost, standardi kvalitete, sigurnost...). Primjeri takvih zahtjeva u Vašem projektu mogu biti: podržani jezici korisničkog sučelja, vrijeme odziva, najveći mogući podržani broj korisnika, podržane web/mobilne platforme, razina zaštite (protokoli komunikacije, kriptiranje...)... Svaki takav zahtjev potrebno je navesti u jednoj ili dvije rečenice.}
			 
			 
			 
	